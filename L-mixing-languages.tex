% Created 2016-09-03 Sat 11:44
\documentclass[10pt, t]{beamer}
\usepackage[utf8]{inputenc}
\usepackage[T1]{fontenc}
\usepackage{fixltx2e}
\usepackage{graphicx}
\usepackage{longtable}
\usepackage{float}
\usepackage{wrapfig}
\usepackage{rotating}
\usepackage[normalem]{ulem}
\usepackage{amsmath}
\usepackage{textcomp}
\usepackage{marvosym}
\usepackage{wasysym}
\usepackage{amssymb}
\usepackage{hyperref}
\tolerance=1000
\usepackage{APC524b}
\hypersetup{colorlinks=true, urlcolor=webgreen}
\let\texttt=\graytt
\let\sout=\code
\let\alert=\textbf
\newcommand{\swig}{\texttt{swig}\xspace}  % swig
\newcommand{\BP}{\texttt{boost::python}\xspace}  % boost::python
\usetheme{default}
\author{Alexey Svyatkovskiy, based on the original slides by Robert Lupton}
\date{22 November 2016}
\title{Calling C/\CPP from Python}
\hypersetup{
  pdfkeywords={},
  pdfsubject={},
  pdfcreator={Emacs 24.5.1 (Org mode 8.2.7c)}}
\begin{document}

\maketitle
\def\wrap{\_wrap}

\section*{Interfacing Python with compiled code}
\label{sec-1}

\begin{frame}[fragile,label=sec-1-1]{Motivation}
 You'd like to be able to write your \CPP and then say things like:
\lstset{language=Python,label= ,caption= ,numbers=none}
\begin{lstlisting}
data = Image(filename)
print data.getWidth()

if data.get(0, 0) < 10:
    debias(data)
\end{lstlisting}

\pause Some things are so much easier in python, and you can reload files while preserving all your data ---
rather like using a debugger instead of \texttt{printf}; python makes a nice high-level interactive debugger for
algorithms and data.

\pause
The authors of debuggers have realised this themselves, and you can now extend
\texttt{gdb} and \texttt{lldb} in python --- but that's a different story.
\end{frame}

%\begin{frame}[fragile,label=sec-1-2]{Motivation: \texttt{SExtractor} lite}
% Here's a partial emulation of a famous astronomical code:
%
%\pause - Read data: 
%\lstset{language=Python,label= ,caption= ,numbers=none}
%\begin{lstlisting}
%	exposure = @\color{red}ExposureF@(fileName)
%\end{lstlisting}
%\pause - Subtract background:
%\lstset{language=Python,label= ,caption= ,numbers=none}
%\begin{lstlisting}
%	im -= @\color{red}makeBackground@(im, bctrl).@\color{red}getImageF@()
%\end{lstlisting}
%\pause - Find detection threshold:
%\lstset{language=Python,label= ,caption= ,numbers=none}
%\begin{lstlisting}
%	stats = @\color{red}makeStatistics@(im, @\color{red}MEANCLIP@ | @\color{red}STDEVCLIP@)
%	threshold = stats.@\color{red}getValue@(@\color{red}MEANCLIP@) + nsigma*stats.@\color{red}getValue@(@\color{red}STDEVCLIP@)
%\end{lstlisting}
%\pause - Smooth image with filter:
%\lstset{language=Python,label= ,caption= ,numbers=none}
%\begin{lstlisting}
%	@\color{red}convolve@(smoothedIm, im, kernel)
%\end{lstlisting}
%\pause - Detect sources:
%\lstset{language=Python,label= ,caption= ,numbers=none}
%\begin{lstlisting}
%	fs = @\color{red}makeFootprintSet@(smoothedIm, threshold, "", npixMin)
%	fs = @\color{red}makeFootprintSet@(fs, grow, isotropic)
%	fs.@\color{red}setMask@(exposure.@\color{red}getMaskedImage@().@\color{red}getMask@(), "DETECTED")
%\end{lstlisting}
%\pause - Measure sources:
%\lstset{language=Python,label= ,caption= ,numbers=none}
%\begin{lstlisting}
%	sources = fs.@\color{red}getFootprints@()
%	measureSources = @\color{red}makeMeasureSources@(exposure)
%	for i in range(len(objects)):
%	    source = @\color{red}Source@()
%  	    measureSources.@\color{red}apply@(source, objects[i])
%\end{lstlisting}
%\emph{N.b.} \CPP is in \begingroup\color{red}red\endgroup; the python is \begingroup\color{blue}blue\endgroup.
%\end{frame}

\begin{frame}[fragile,label=sec-1-3]{Tools to bind C/\CPP/Fortran to python}
 There is a large variety of solutions available to the problem of binding C/\CPP to python; the 
common ones are:
\begin{itemize}
\item \texttt{cython}
\item \texttt{ctypes} module and attribute
\item \BP
\item \swig
\item hand-crafted code using the python C API, \emph{CPython}
\end{itemize}

\pause
There's also \texttt{PyPy} (\url{http://pypy.org}) which replaces the traditional \texttt{C} implementation of \texttt{python} with
a \textit{J}ust \textit{I}n \textit{T}ime compiler (\emph{JIT}) that provides significant speedups.
\pause
\end{frame}

\section*{cython}
\label{sec-2}
\begin{frame}[fragile,label=sec-2-1]{\texttt{cython}}
 \texttt{cython} lets you write python and speed up some critical part of the code.
\pause 
\texttt{cython} can also be used to bind C/\CPP to python --- a topic I'll return to later.

\pause
For example, I have a file \alert{hello.pyx}:
\snippetFile[python]{src/hello.pyx}

\lstset{language=Python,label= ,caption= ,numbers=none}
\begin{lstlisting}
def speak(name):
    print("Hello %s" % name)
Hello Alexey
\end{lstlisting}
\pause
\lstset{language=text,label= ,caption= ,numbers=none}
\begin{lstlisting}
>>> import pyximport
>>> pyximport.install()
>>> import hello
>>> hello.speak("Alexey")
Hello Alexey
\end{lstlisting}
\pause
Trivia: it's "pyx" is because \texttt{cython} is a fork of an old project called \texttt{pyrex}.

\pause
\lstset{language=text,label= ,caption= ,numbers=none}
\begin{lstlisting}
$ wc -l ~/.pyxbld/temp.linux-x86_64-2.7/pyrex/hello.c 
1792 /home/alexeys/.pyxbld/temp.linux-x86_64-2.7/pyrex/hello.c
\end{lstlisting}

That innocent 2-line \texttt{cython} script generated 1792 lines of C.

\begin{block}{\texttt{cython} website}
\url{http://cython.org}
\end{block}
\end{frame}

\begin{frame}[fragile,label=sec-2-2]{Building \texttt{cython} extensions}
 We just saw \texttt{pyximport} as a simple way to build cython extensions.  
\pause
An alternative is to use python's \texttt{distutils}; this requires a \alert{setup.py} file along these lines:
\lstset{language=Python,label= ,caption= ,numbers=none}
\begin{lstlisting}
from distutils.core import setup
from distutils.extension import Extension
from Cython.Distutils import build_ext

setup(
    cmdclass = {'build_ext': build_ext},
    ext_modules = [Extension("hello", ["hello.pyx"])]
)
\end{lstlisting}
\pause
which can be invoked as
\lstset{language=text,label= ,caption= ,numbers=none}
\begin{lstlisting}
$ python setup.py build_ext --inplace
\end{lstlisting}
\end{frame}

\begin{frame}[fragile,label=sec-2-3]{\texttt{cython} timings}
 Stealing an example from the \texttt{cython} documentation
(\url{http://docs.cython.org/src/quickstart/cythonize.html}),
\lstset{language=Python,label= ,caption= ,numbers=none}
\begin{lstlisting}
def f(x):
    return x**2-x

def integrate_f(a, b, N):
    s = 0
    dx = (b-a)/N
    for i in range(N):
        s += f(a+i*dx)
    return s * dx
\end{lstlisting}

\pause
We can time this with:
\lstset{language=Python,label= ,caption= ,numbers=none}
\begin{lstlisting}
>>> import timeit; import hello
>>> t = timeit.Timer("import hello; hello.integrate_f(0, 10.0, 10**7)")
>>> print "%.2fs" % t.timeit(1)
5.46s
\end{lstlisting}

\pause
If I use \texttt{pyximport} to import (\emph{i.e.} convert to C, compile, and dynamically load) that file,
 the same test takes 3.85s
\end{frame}

\begin{frame}[fragile,label=sec-2-4]{Using \texttt{cython}'s \texttt{cdef} on variables}
 The first thing to do is to declare some variable's types with \texttt{cdef}:
\lstset{language=Python,label= ,caption= ,numbers=none}
\begin{lstlisting}
def f1(double x):
    return x**2-x

def integrate_f1(double a, double b, int N):
    cdef int i
    cdef double s, dx
    s = 0
    dx = (b-a)/N
    for i in range(N):
        s += f1(a+i*dx)
    return s * dx
\end{lstlisting}
\pause This runs in 1.07s

\begin{itemize}
\item Since the iterator variable \texttt{i} is typed with C semantics, the for-loop will be compiled to pure C code.
\item Typing \texttt{a}, \texttt{s} and \texttt{dx} is important as they are involved in arithmetic within the for-loop
\item Typing \texttt{b} and \texttt{N} makes less of a difference
\end{itemize}
\end{frame}

\begin{frame}[fragile,label=sec-2-5]{Using \texttt{cython}'s \texttt{cdef} on functions}
 Next we can change \texttt{f} to use the C calling sequence, \emph{i.e.} passing C variables (in registers?) rather than
converting to and from python objects:
\lstset{language=Python,label= ,caption= ,numbers=none}
\begin{lstlisting}
def f2(double x) except? -2:
    return x**2-x

def integrate_f2(double a, double b, int N):
    cdef int i
    cdef double s, dx
    s = 0
    dx = (b-a)/N
    for i in range(N):
        s += f2(a+i*dx)
    return s * dx
\end{lstlisting}
\pause This runs in 0.05s

\pause
\emph{N.b.} This version of \texttt{f} is of course only callable from \texttt{cython} code, not vanilla \texttt{python}.

\pause
The \sout{except? -2} bit says "If you return -2 check if an error occurred, and maybe throw an exception".
\end{frame}

\begin{frame}[fragile,label=sec-2-6]{Summary of timings}
 \begin{itemize}
\item python
\end{itemize}
\begin{center}
\begin{tabular}{ll}
\texttt{f} & 5.46s\\
\end{tabular}
\end{center}
\begin{itemize}
\item cython
\end{itemize}
\begin{center}
\begin{tabular}{ll}
\texttt{f} & 3.85s\\
\texttt{f1} & 1.07s\\
\texttt{f2} & 0.05s\\
\end{tabular}
\end{center}


Clearly using \texttt{cython} can provide dramatic speedups for critical sections of code.
\end{frame}

\section*{ctypes}
\label{sec-3}
\begin{frame}[fragile,label=sec-3-1]{\texttt{ctypes}}

if your use case is calling C library functions or system calls, you should consider using the \texttt{ctypes} module
rather than writing custom C code. Not only does \texttt{ctypes} let you write Python code to interface with C code,
but it is more portable between implementations of Python than writing and compiling an extension module

\pause
\texttt{ctypes} is a builtin part of python (as of python 2.5)
\url{http://docs.python.org/library/ctypes.html\#module-ctypes}

\pause
\texttt{ctypes} provides a way to access functions (and other symbols) in sharable object libraries.
\pause
Despite this official status, my reading is that \texttt{cython} is slowly taking over from \texttt{ctypes} 
for wrapping scientific code.
\end{frame}

\begin{frame}[fragile,label=sec-3-2]{Using \texttt{ctypes} to access C standard library}
 \snippetFile[python]{src/ctypes_demo.py}

Loading a sharable library (in this case \texttt{libc}) is easy:
\lstset{language=Python,label= ,caption= ,numbers=none}
\begin{lstlisting}
import ctypes
libc = ctypes.CDLL("libc.dylib")
\end{lstlisting}

\pause
In general you may not know the file to load, so you can use something like
\lstset{language=Python,label= ,caption= ,numbers=none}
\begin{lstlisting}
import ctypes.util
libc = ctypes.util.find_library("c")
libc = ctypes.CDLL(libc)
\end{lstlisting}

\pause
It's easy to use too:
\lstset{language=Python,label= ,caption= ,numbers=none}
\begin{lstlisting}
libc.system("echo hello world")
\end{lstlisting}
\end{frame}

%\begin{frame}[fragile,label=sec-3-3]{\texttt{ctypes}' limitations}
% The \texttt{ctypes} is interface is \alert{very} low level.  For example,
%\lstset{language=Python,label= ,caption= ,numbers=none}
%\begin{lstlisting}
%strchr = libc.strchr
%print strchr("abcdef",ord("d"))
%\end{lstlisting}
%prints \texttt{8059980} (the address of the string \texttt{"def"})

%\pause
%The problem is that \texttt{ctypes} doesn't know \texttt{strchr}'s return type, but we can tell it:
%\lstset{language=Python,label= ,caption= ,numbers=none}
%\begin{lstlisting}
%strchr.restype = ctypes.c_char_p
%print strchr("abcdef",ord("d"))
%\end{lstlisting}
%prints \texttt{'def'}.

%\pause
%Why should I have to type \texttt{ord("d")}, which converts \texttt{d} to an \texttt{char}?
%\pause I don't:
%\lstset{language=Python,label= ,caption= ,numbers=none}
%\begin{lstlisting}
%strchr.argtypes = [ctypes.c_char_p,ctypes.c_char]
%print strchr("abcdef","d")
%\end{lstlisting}
%prints \texttt{'def'} and 
%\lstset{language=Python,label= ,caption= ,numbers=none}
%\begin{lstlisting}
%try:
%    strchr("abcdef","def")
%except Exception e:
%    print e
%\end{lstlisting}
%raises a \texttt{ctypes.ArgumentError} exception
%\end{frame}

%\begin{frame}[fragile,label=sec-3-4]{\texttt{ctypes}' limitations}
% Without \texttt{argtypes},
%\lstset{language=Python,label= ,caption= ,numbers=none}
%\begin{lstlisting}
%print libc.strlen(1)
%\end{lstlisting}
%crashes python. 
%\pause
%Fortunately,
%\lstset{language=Python,label= ,caption= ,numbers=none}
%\begin{lstlisting}
%libc.system.argtypes = [ctypes.c_char_p]
%try:
%    print libc.strlen(1)
%except Exception e:
%    print e
%\end{lstlisting}
%raises the expected \texttt{ctypes.ArgumentError} exception.
%\pause
%Unfortunately,
%\lstset{language=Python,label= ,caption= ,numbers=none}
%\begin{lstlisting}
%print libc.strlen("abc","def")
%\end{lstlisting}
%quietly returns \texttt{3}
%\end{frame}



\begin{frame}[fragile,label=sec-3-5]{Defining custom types with \texttt{ctypes}}
The ctypes module of the Python standard library provides definitions of fundamental data types that can be passed to C programs. For example a C double type:
\lstset{language=Python,label= ,caption= ,numbers=none}
\begin{lstlisting}
import ctypes as C
x = C.c_double(2.71828)
\end{lstlisting}
\pause
And here is the pointer type to double:
\lstset{language=Python,label= ,caption= ,numbers=none}
\begin{lstlisting}
xp = C.POINTER(C.c_double)(); 
xp.contents = x
\end{lstlisting}

\pause
Which gives:
\lstset{language=Python,label= ,caption= ,numbers=none}
\begin{lstlisting}
>>> print(xp)
>>> <__main__.LP_c_double object at 0x7fb63427c9e0>
>>> print(x)
>>> c_double(2.71828)
\end{lstlisting}

\pause
Array types can be created by multiplying a ctype by a positive integer, e.g.:
\lstset{language=Python,label= ,caption= ,numbers=none}
\begin{lstlisting}
ylist = [1.,2.3,4.,5.]
n = len(ylist)
y = (C.c_double*n)()
y[:] = ylist
#or simply
y = (C.c_double*n)(*ylist)
\end{lstlisting}
\end{frame}

\begin{frame}[fragile,label=sec-3-6]{\texttt{ctypes}: a complete example I}
Let us start by writing some C code. The dot product of two vectors for instance:
\lstset{language=C,label= ,caption= ,numbers=none}
\begin{lstlisting}
double dot_product(double v[], double u[], int n)
{
    double result = 0.0;
    for (int i = 0; i < n; i++)
        result += v[i]*u[i];
    return result;
}
\end{lstlisting}
\pause
Next we compile it, and build a shared object:
\lstset{language=C,label= ,caption= ,numbers=none}
\begin{lstlisting}
gcc -c -Wall -Werror -fpic my_dot.c 
gcc -shared -o my_dot.so my_dot.o
\end{lstlisting}
\end{frame}

\begin{frame}[fragile,label=sec-3-7]{\texttt{ctypes}: a complete example II}
We have already seen that the \texttt{ctypes} module has a utility subpackage to assist in locating a dynamically-loaded library:
\lstset{language=C,label= ,caption= ,numbers=none}
\begin{lstlisting}
import ctypes.util
C.util.find_library('my_dot')
myDL = C.CDLL('./my_dot.so')
\end{lstlisting}
\pause
Here is a full example:
\lstset{language=C,label= ,caption= ,numbers=none}
\begin{lstlisting}
from ctypes import CDLL, c_int, c_double
mydot = CDLL('./my_dot.so').dot_product
def dot(vec1, vec2): # vec1, vec2 are Python lists
    n = len(vec1)
    mydot.restype = c_double
    return mydot((c_double*n)(*vec1), (c_double*n)(*vec2), c_int(n))

vec1 = [x for x in range(1000000)]
vec2 = [x for x in range(1000000)]
\end{lstlisting}
\pause
Warning: if you use the extension .so for the name of a file, do not make its stem the same as a .py file in the same directory, e.g., do not have both a funcs.py and a funcs.so.
\end{frame}


%\begin{frame}[fragile,label=sec-5-5]{Returning a \texttt{struct}}
% Returning to an earlier example, we had
%\lstset{language=C,label= ,caption= ,numbers=none}
%\begin{lstlisting}
%struct {
%    char *first;	 /* first name */
%    char *last; 	 /* last name */
%    int number;
%} Noddy;
%\end{lstlisting}

%\pause
%To return one of these via \texttt{ctypes} I need to define (on the python side)
%\lstset{language=Python,label= ,caption= ,numbers=none}
%\begin{lstlisting}
%  import ctypes
%  
%  class Noddy(ctypes.Structure):
%      _fields_ = [
%          ('first', ctypes.c_char_p),
%          ('last', ctypes.c_char_p),
%          ('number', ctypes.c_int),
%          ]
%      
%  function.restype = ctypes.POINTER(Noddy)
%\end{lstlisting}

%\pause
%I obviously need to keep this is sync with the \CPP version.  

%\pause
%I don't know how to return a \texttt{class} with virtual functions (and thus a \texttt{vtbl}); the
%\texttt{c} in \texttt{ctypes} really does seem to stand for C, not \CPP.
%\end{frame}

\begin{frame}[fragile,label=sec-3-8]{\texttt{ctypes} and numpy}
 Here's an example taken from \href{http://www.scipy.org/Cookbook/Ctypes}{the ctypes manual}

\lstset{language=Python,label= ,caption= ,numbers=none}
\begin{lstlisting}
  import numpy
  import ctypes
  
  # Extract desired information from libfoo.so [or libfoo.dylib]
  _foo = numpy.ctypeslib.load_library('libfoo', '/my/working/directory')
  _foo.bar.restype = ctypes.c_int
  _foo.bar.argtypes = [ctypes.POINTER(ctypes.c_double), ctypes.c_int]
  
  def bar(x):
      """Wrapper to call C function 'bar' nicely from python"""
      return _foo.bar(x.ctypes.data_as(ctypes.POINTER(ctypes.c_double)), len(x))
  
  x = numpy.random.randn(10)
  n = bar(x)
\end{lstlisting}

Note that \texttt{numpy} arrays provide \texttt{.ctypes} to extract the information that \texttt{ctypes} needs; there's also \emph{e.g.}
\texttt{x.ctypes.shape[:3]}
\end{frame}

\section*{cython and C/\CPP}
\label{sec-4}
\begin{frame}[fragile,label=sec-4-1]{\texttt{cython} and C/\CPP}
 \texttt{cython} can also be used to bind C/CPP to python.

\pause
For example, 
\lstset{language=C,label= ,caption= ,numbers=none}
\begin{lstlisting}
// Return the greatest common divisor of a and b
int gcd(int a, int b);
\end{lstlisting}
(the details are left to your imagination).  
\pause
To use this from python using \texttt{cython}, we need an interface file \texttt{simple.pyx}:
\lstset{language=Python,label= ,caption= ,numbers=none}
\begin{lstlisting}
cdef extern from "gcd.h":
    int c_gcd "gcd" (int a, int b)

def gcd(int a, int b):
    return c_gcd(a, b)
\end{lstlisting}
\end{frame}

\begin{frame}[fragile,label=sec-4-2]{\texttt{cython} and C/\CPP}
 We next need to build the glue layer; the Makefile looks like:
\lstset{language=sh,label= ,caption= ,numbers=none}
\begin{lstlisting}
simple.so: simple.pyx gcd.c gcd.h
        python setup.py build_ext --inplace
\end{lstlisting}
\pause
with
\lstset{language=Python,label= ,caption= ,numbers=none}
\begin{lstlisting}
from distutils.core import setup
from distutils.extension import Extension
from Cython.Distutils import build_ext

source_files = ["simple.pyx", "gcd.c"]

ext_modules = [Extension(
    name="simple",
    sources=source_files,
    # extra_objects=["fc.o"],  # if you compile fc.cpp separately
    extra_compile_args = "-std=c99".split(),
    # extra_link_args = "...".split()
    )]

setup(
    cmdclass = {'build_ext': build_ext},
    ext_modules = ext_modules
)
\end{lstlisting}
\end{frame}

\begin{frame}[fragile,label=sec-4-3]{\texttt{cython} and C/\CPP}
 We can then triumphantly type
\lstset{language=Python,label= ,caption= ,numbers=none}
\begin{lstlisting}
import simple

x = 52
y = 65
z = simple.gcd(x,y)
print "The gcd of %d and %d is %d" % (x,y,z)
\end{lstlisting}
\pause
and discover that
\lstset{language=sh,label= ,caption= ,numbers=none}
\begin{lstlisting}
The gcd of 52 and 65 is 13
\end{lstlisting}
\end{frame}

\begin{frame}[fragile,label=sec-4-4]{\texttt{cython} and \texttt{numpy}}
 \lstset{language=Python,label= ,caption= ,numbers=none}
\begin{lstlisting}
# -*- python -*-
#
# Import the bits we need from C
#
cdef extern from "vector_ops.h":
    void c_scalar_multiply "scalar_multiply" (double alpha,
                                              double *x, double *z, long n)
    void c_vector_add "vector_add" (double *x, double *y, double *z, long n)
#
# Define wrapper functions to be used from python
#
import numpy as np
cimport numpy as np

def scalar_multiply(double alpha, np.ndarray[np.double_t,ndim=1] x):
    cdef long n = x.shape[0]
    cdef np.ndarray z = np.empty(n, dtype=np.double)
    c_scalar_multiply(alpha, <double *> x.data, <double *> z.data, n)
    return z

def vector_add(np.ndarray[np.double_t,ndim=1] x,
               np.ndarray[np.double_t,ndim=1] y):
    cdef long n = x.shape[0]
    cdef np.ndarray z = np.empty(n, dtype=np.double)
    c_vector_add(<double *> x.data, <double *> y.data, <double *> z.data, n)
    return z
\end{lstlisting}
\end{frame}

\begin{frame}[fragile,label=sec-4-5]{\texttt{cython} and \texttt{numpy}}
 Building is very similar to the previous example; here's the diff for \file{setup.py}:
\lstset{language=sh,label= ,caption= ,numbers=none}
\begin{lstlisting}
$ diff numpy/setup.py simple/setup.py 
1d0
< import numpy
6c5
< source_files = ["vec_ops.pyx", "vector_ops.c"]
---
> source_files = ["simple.pyx", "gcd.c"]
9c8
<     name="vec_ops",
---
>     name="simple",
12d10
<     include_dirs = [numpy.get_include()],
\end{lstlisting}
\end{frame}

\begin{frame}[fragile,label=sec-4-6]{\texttt{cython} and \texttt{numpy}}
 Using our new extension is as easy as:
\lstset{language=Python,label= ,caption= ,numbers=none}
\begin{lstlisting}
import vec_ops, numpy as np

alpha, x = 2.1, np.array([1.,2.,3.])
y = vec_ops.scalar_multiply(alpha, x)
print alpha, "times", x, "is", y

z = vec_ops.vector_add(x, y)
print x, "plus", y, "is", z
\end{lstlisting}
\pause
which prints
\lstset{language=Python,label= ,caption= ,numbers=none}
\begin{lstlisting}
2.1 times [ 1.  2.  3.] is [ 2.1  4.2  6.3]
[ 1.  2.  3.] plus [ 2.1  4.2  6.3] is [ 3.1  6.2  9.3]
\end{lstlisting}
\end{frame}
\begin{frame}[fragile,label=sec-4-7]{Digression: compiler warnings}
 there are some compiler warnings from the machine-generated code; \emph{e.g.}
\lstset{language=sh,label= ,caption= ,numbers=none}
\begin{lstlisting}
vec_ops.c: In function '__pyx_pf_7vec_ops_2vector_add':
vec_ops.c:1363: warning: implicit conversion shortens 64-bit value into a 32-bit value
vec_ops.c: In function '__Pyx_GetBuffer':
vec_ops.c:4649: warning: unused variable 'getbuffer_cobj'
\end{lstlisting}
\pause
Some of these are avoidable: in \file{vector\_ops.h} we see:
\lstset{language=C,label= ,caption= ,numbers=none}
\begin{lstlisting}
void scalar_multiply(double alpha, const double *x, double *z, int n);
\end{lstlisting}
\pause
(should be \code{long})
\end{frame}

\begin{frame}[fragile,label=sec-4-8]{Digression: compiler warnings}
 Some can't be avoided without delving into \texttt{cython} internals:
\lstset{language=C,label= ,caption= ,numbers=none}
\begin{lstlisting}
static int __Pyx_GetBuffer(PyObject *obj, Py_buffer *view, int flags) 
{
    PyObject *getbuffer_cobj;
  
    if (PyObject_TypeCheck(obj, __pyx_ptype_5numpy_ndarray)) 
       return __pyx_pw_5numpy_7ndarray_1__getbuffer__(obj, view, flags);

#if PY_VERSION_HEX < 0x02060000
    if (obj->ob_type->tp_dict &&
        (getbuffer_cobj = PyMapping_GetItemString(obj->ob_type->tp_dict,
                                                  "__pyx_getbuffer"))) {
...
\end{lstlisting}
\pause
That would have been better written as:
\lstset{language=C,label= ,caption= ,numbers=none}
\begin{lstlisting}
static int __Pyx_GetBuffer(PyObject *obj, Py_buffer *view, int flags) 
{
    if (PyObject_TypeCheck(obj, __pyx_ptype_5numpy_ndarray)) 
       return __pyx_pw_5numpy_7ndarray_1__getbuffer__(obj, view, flags);

#if PY_VERSION_HEX < 0x02060000
    PyObject *getbuffer_cobj;
    if (obj->ob_type->tp_dict &&
        (getbuffer_cobj = PyMapping_GetItemString(obj->ob_type->tp_dict,
                                                  "__pyx_getbuffer"))) {
...
\end{lstlisting}
\end{frame}

\section*{\BP}
\label{sec-5}
\begin{frame}[label=sec-5-1]{\BP}
\begin{block}{\BP}
\url{http://www.boost.org/doc/libs/1_57_0/libs/python/doc/index.html}

\url{http://www.boost.org/doc/libs/1_57_0/libs/python/doc/tutorial/doc/html/index.html}

\pause
There is also
\url{http://wiki.python.org/moin/boost.python}
\end{block}
\end{frame}

\begin{frame}[fragile,label=sec-5-2]{\texttt{greet} in \BP}
 \snippetFile[C++]{src/speak.cc}

(This is based on
\url{http://www.boost.org/doc/libs/1_57_0/libs/python/doc/tutorial/doc/html/index.html})

Let's return to an old friend, \texttt{greet}.
\pause
We first need \texttt{greet} itself:

\lstset{language=C,label= ,caption= ,numbers=none}
\begin{lstlisting}
std::string greet(const std::string &str="world") {
   return "hello " + str;
}
\end{lstlisting}

(we could \texttt{\#include "greet.h"} instead).
\pause
To bind this using \BP, we need to include the proper header:

\lstset{language=C,label= ,caption= ,numbers=none}
\begin{lstlisting}
#include "boost/python.hpp"
\end{lstlisting}

\pause
and then
\lstset{language=C,label= ,caption= ,numbers=none}
\begin{lstlisting}
BOOST_PYTHON_MODULE(speak)
{
  using namespace boost::python;
  def("greet", greet);
}
\end{lstlisting}
\pause
After building our extension \texttt{speak}, we can say
\lstset{language=Python,label= ,caption= ,numbers=none}
\begin{lstlisting}
>>> import speak
>>> print speak.greet("class")
hello class
\end{lstlisting}
\end{frame}

\begin{frame}[fragile,label=sec-5-3]{Building with \BP}
 Building \BP extensions is easy, \emph{e.g.} on Della cluster:
\lstset{language=textc,label= ,caption= ,numbers=none}
\begin{lstlisting}
$ module load boost/1.55.0 
$ g++ -o speak.os -c -fPIC -g -Wall \
        -I/usr/include/boost -I/usr/include/python2.7 speak.cc
$ g++ -o speak.so -shared speak.os -lpython2.7 -lboost_python
\end{lstlisting}
\pause
If you need to install \texttt{boost} on your own machine look at the \emph{getting started} section of 
\url{http://www.boost.org} or go a-googling.

\pause
\emph{N.b.} the \texttt{boost} documentation recommends building \alert{speak.cc} with \texttt{bjam};  ignore this
advice.  With modernish versions of \texttt{boost} there's no need to lie on that bed of nails.
\end{frame}

\begin{frame}[fragile,label=sec-5-4]{Overloaded functions}
 In
\lstset{language=C,label= ,caption= ,numbers=none}
\begin{lstlisting}
std::string greet(const std::string &str="world") {
   return "hello " + str;
}
\end{lstlisting}
we provided a default value for \texttt{str}, so this should work:
\lstset{language=Python,label= ,caption= ,numbers=none}
\begin{lstlisting}
>>> import speak
>>> print speak.greet()
\end{lstlisting}
\pause
but in reality:
\lstset{language=Python,label= ,caption= ,numbers=none}
\begin{lstlisting}
>>> print speak.greet()
Traceback (most recent call last):
  File "<stdin>", line 1, in <module>
Boost.Python.ArgumentError: Python argument types in
    speak.greet()
did not match C++ signature:
    greet(std::string)
\end{lstlisting}
(but at least that's a good error message)
\end{frame}

\begin{frame}[fragile,label=sec-5-5]{Overloaded functions}
 \BP can handle this, but it takes a bit more work:

\lstset{language=Python,label= ,caption= ,numbers=none}
\begin{lstlisting}
BOOST_PYTHON_FUNCTION_OVERLOADS(greet_s_overloads, greet, 0, 1)
BOOST_PYTHON_MODULE(speak)
{
  using namespace boost::python;
  def("greet", greet, greet_s_overloads());
}
\end{lstlisting}

\pause
The \texttt{BOOST\_PYTHON\_FUNCTION\_OVERLOADS} says that \texttt{greet} takes from \texttt{0} to \texttt{1} arguments, and
\BP does the rest, generating the function \texttt{greet\_s\_overloads}.
\end{frame}

\begin{frame}[label=sec-5-6]{Overloaded functions}
If we also add
\lstset{language=C,label= ,caption= ,numbers=none}
\begin{lstlisting}
std::string greet(int i) {
   std::ostringstream ss;
   ss << "Hello " << i;
   return ss.str();
}
\end{lstlisting}
and
\lstset{language=C,label= ,caption= ,numbers=none}
\begin{lstlisting}
std::string greet(const std::string &str, int nrepeat) {
   std::ostringstream ss;
   ss << "Hello";
   for (int i = 0; i != nrepeat; ++i) ss << " " << str;
   return ss.str();
}
\end{lstlisting}
\pause
We need to tell \BP:
\lstset{language=Python,label= ,caption= ,numbers=none}
\begin{lstlisting}
std::string (*greet_i)(int) = &greet;
std::string (*greet_s)(const std::string &, int i) = &greet;

BOOST_PYTHON_FUNCTION_OVERLOADS(greet_s_overloads, greet, 0, 2)
BOOST_PYTHON_MODULE(speak)
{
   using namespace boost::python;
   def("greet", greet_i);
   def("greet", greet_s, greet_s_overloads());
}
\end{lstlisting}
\end{frame}

\begin{frame}[fragile,label=sec-5-7]{\texttt{classes}}
 Consider
\lstset{language=Python,label= ,caption= ,numbers=none}
\begin{lstlisting}
class X {
public:
   void set(const std::string& msg) { msg_ = msg; }
   std::string greet() { return "hello " + msg_; }
private:
   std::string msg_;
};
\end{lstlisting}

\pause
The \BP incantation is:
\lstset{language=Python,label= ,caption= ,numbers=none}
\begin{lstlisting}
BOOST_PYTHON_MODULE(speak)
{
   using namespace boost::python;
   class_<X>("X")
      .def("greet", &X::greet)
      .def("set", &X::set);
}
\end{lstlisting}

\pause
after which:
\lstset{language=Python,label= ,caption= ,numbers=none}
\begin{lstlisting}
>>> import speak
>>> X = speak.X()
>>> x.set("Alexey")
>>> print x.greet()
hello Alexey
\end{lstlisting}
\end{frame}

\begin{frame}[fragile,label=sec-5-8]{Data members in \texttt{classes}}
 Consider
\lstset{language=Python,label= ,caption= ,numbers=none}
\begin{lstlisting}
struct Y {
   int i;
};
\end{lstlisting}

\pause
We need to say what sort of access we need to \texttt{i}, \emph{e.g.}
\lstset{language=Python,label= ,caption= ,numbers=none}
\begin{lstlisting}
BOOST_PYTHON_MODULE(speak)
{
   using namespace boost::python;
   class_<Y>("Y")
    .def_readwrite("i", &Y::i);
}
\end{lstlisting}

\pause
and:
\lstset{language=Python,label= ,caption= ,numbers=none}
\begin{lstlisting}
>>> import speak
>>> y = speak.Y()
>>> y.i
0
>>> y.i = 10
>>> y.i
10
\end{lstlisting}
\end{frame}

\begin{frame}[fragile,label=sec-5-9]{\BP and the \emph{STL}}
STL is well support in \BP as well:
\lstset{language=C,label= ,caption= ,numbers=none}
\begin{lstlisting}
#include "boost/python.hpp"
#include "boost/python/suite/indexing/vector_indexing_suite.hpp"
BOOST_PYTHON_MODULE(speak)
{
   using namespace boost::python;
   class_<std::vector<double> >("vectorDouble")
   .def(vector_indexing_suite<std::vector<double> >());
}
\end{lstlisting}

\pause
\lstset{language=Python,label= ,caption= ,numbers=none}
\begin{lstlisting}
>>> import speak
>>> v = speak.vectorDouble()
>>>  v.append(10); v.append(100)
>>> print len(v), [x for x in v]
2 [10.0, 100.0]
\end{lstlisting}
\end{frame}

\begin{frame}[fragile,label=sec-5-10]{Functions that take \texttt{std::vector<double>}}
 If we define
\lstset{language=C,label= ,caption= ,numbers=none}
\begin{lstlisting}
#include <iostream>
#include <iterator>
...
template<typename T>
void print_vector(const std::vector<T>& v) {
   std::copy(v.begin(), v.end(),
   std::ostream_iterator<T>(std::cout, " "));
   std::cout << std::endl;
}
\end{lstlisting}

\pause
and add
\lstset{language=Python,label= ,caption= ,numbers=none}
\begin{lstlisting}
def("print_vector", print_vector<double>);
\end{lstlisting}

\pause
we can say:
\lstset{language=Python,label= ,caption= ,numbers=none}
\begin{lstlisting}
>>> speak.print_vector(v)
10 100
\end{lstlisting}
\end{frame}

\section*{\protect\swig}
\label{sec-6}
\begin{frame}[label=sec-6-1]{What is \swig?}
The \textit{S}implified \textit{W}rapper and \textit{I}nterface \textit{G}enerator,
(\swig;
\url{http://www.swig.org}
) is a way of \alert{automatically} generating code that interfaces C or \CPP to \{
C\#,
Guile,
Java,
Lua,
Modula 3,
Mzscheme,
Ocaml,
Octave,
Perl,
PHP4,
PHP5,
Pike,
Python,
R (aka GNU S),
Ruby,
Lisp S-Expressions,
Tcl,
Common Lisp / UFFI,
XML
\}
that handles all of these concerns, and more.

\begin{block}{The \swig online documentation}
\url{http://www.swig.org/Doc3.0/index.html}

The latest version as of 2016-06-12 is 3.0.10
\end{block}
\end{frame}

\begin{frame}[fragile,label=sec-6-2]{\emph{Hello World}}
 Consider a couple of source files, \alert{hello.h}:
\lstset{language=c++,label= ,caption= ,numbers=none}
\begin{lstlisting}
#if !defined(HELLO_H)
#define HELLO_H 1

#include <string>

void speak(std::string const& str);
#endif
\end{lstlisting}
and \alert{hello.c}:
\lstset{language=c++,label= ,caption= ,numbers=none}
\begin{lstlisting}
#include <iostream>
#include "hello.h"

void speak(std::string const& str)
{
    std::cout << "Alexey says: " << str << std::endl;
}
\end{lstlisting}
\end{frame}

\begin{frame}[fragile,label=sec-6-3]{\emph{Hello World}}
 With this \swig interface file, \alert{hello.i}:
\lstset{language=C,label= ,caption= ,numbers=none}
\begin{lstlisting}
%module hello

%{
#include "hello.h"
%}

%include "hello.h"
\end{lstlisting}
\end{frame}

\begin{frame}[fragile,label=sec-6-4]{\emph{Hello World}}
 and after running make:
\lstset{language=textc,label= ,caption= ,numbers=none}
\begin{lstlisting}
$ swig -o hello_wrap.cc -c++ -python hello.i
$ g++ -o hello.os -c -fPIC -I/usr/include/python2.7 hello.cc
$ g++ -o hello_wrap.os -c -fPIC -I/usr/include/python2.7 hello_wrap.cc
$ g++ -o _hello.so -bundle -flat_namespace hello_wrap.os hello.os \
        -L/Library/Python/2.7/site-packages -lpython
\end{lstlisting}

\gdef\hello{\_hello}
\pause
\emph{i.e.}
\begin{itemize}
\item Run \swig to generate \alert{hello\wrap.cc} (and also, as we shall see, \alert{hello.py})
\item Compile \alert{hello.cc} and \alert{hello\wrap.cc} to create object files \textbf{*.os}
    with the usual compile-python boilerplate.
\item Build a loadable module \alert{\hello.so} using the usual os/x dylib boilerplate.
\end{itemize}

\pause
I can start python and import my new module
\pause
\lstset{language=Python,label= ,caption= ,numbers=none}
\begin{lstlisting}
>>> import hello
>>> hello.speak("hello world")
Alexey says: hello world
\end{lstlisting}
\end{frame}

\begin{frame}[label=sec-6-5]{How does \swig earn its keep?}
What did \swig actually do?  It wrote two files, \alert{hello\wrap.cc} (which we just compiled)
and \alert{hello.py}.  When we started python there were two possible files we could import:
\alert{\hello.so} and \alert{hello.py}. The former is created from \alert{hello\wrap.cc} --- a
file that you really don't want to examine.
\end{frame}

\begin{frame}[fragile,label=sec-6-6]{\alert{\wrap.cc} files}
 You may not want to, but\ldots{}
\pause
A small and simplified part of \alert{hello\wrap.cc}'s 4k lines reads:
\lstset{language=C,label= ,caption= ,numbers=none}
\begin{lstlisting}
SWIGINTERN PyObject *_wrap_speak(PyObject *SWIGUNUSEDPARM(self),
                                 PyObject *args) {
  std::string *arg1 = 0 ;
  int res1 = SWIG_OLDOBJ ;
  PyObject * obj0 = 0 ;
  
  if (!PyArg_ParseTuple(args,(char *)"O:speak",&obj0)) SWIG_fail;

  std::string *ptr = (std::string *)0;
  res1 = SWIG_AsPtr_std_string(obj0, &ptr);
  if (!SWIG_IsOK(res1)) {
    SWIG_exception_fail(SWIG_ArgError(res1), "in method 'speak' "
                        "argument 1 of type 'std::string const &'"); 
  }
  arg1 = ptr;

  @\color{red}speak((std::string const \&)*arg1);@

  if (SWIG_IsNewObj(res1)) delete arg1;
  return SWIG_Py_Void();
fail:
  if (SWIG_IsNewObj(res1)) delete arg1;
  return NULL;
}
\end{lstlisting}
\end{frame}

\begin{frame}[fragile,label=sec-6-7]{Explanations}
 What's going on?  This is within an \texttt{extern "C"} block, so it defines a callable function
\texttt{\_wrap\_speak()} that checks the argument type and calls \texttt{speak()}.

\pause
Let's read some more of \alert{hello\wrap.cc}:
\lstset{language=c++,label= ,caption= ,numbers=none}
\begin{lstlisting}
static PyMethodDef SwigMethods[] = {
	 { (char *)"SWIG_PyInstanceMethod_New",
          (PyCFunction)SWIG_PyInstanceMethod_New, METH_O, NULL},
	 @\color{red}{ (char *)"speak", \_wrap\_speak, METH\_VARARGS, NULL}@,
	 { NULL, NULL, 0, NULL }
};
...
void
SWIG_init(void) {
  PyObject *m, *d;

  m = Py_InitModule((char *) SWIG_name, SwigMethods);
  d = PyModule_GetDict(m);
  
  SWIG_InitializeModule(0);
  SWIG_InstallConstants(d,swig_const_table);
}
\end{lstlisting}

\emph{I.e.} we define a module \texttt{\_hello} that knows the command \texttt{speak}.
\end{frame}

\begin{frame}[fragile,label=sec-6-8]{Syntactic sugar}
 However, in python we said \texttt{hello.speak("hello world")}, not
\texttt{\_hello.speak("hello world")}.

\pause

Enter \alert{hello.py}.  In this case it's more-or-less trivial:
\lstset{language=Python,label= ,caption= ,numbers=none}
\begin{lstlisting}
import _hello
...
speak = _hello.speak
\end{lstlisting}
\end{frame}

\begin{frame}[fragile,label=sec-6-9]{\swig's \texttt{\%inline} directive}
 I could have avoided the extra files with:
\lstset{language=c++,label= ,caption= ,numbers=none}
\begin{lstlisting}
// -*- c++ -*-
%module hello_inline
%include "std_string.i"

%{
#include <iostream>
%}

%inline %{
void speak(std::string const& str)
{
    std::cout << "Alexey says: " << str << std::endl;
}
%}
\end{lstlisting}
\end{frame}

\begin{frame}[fragile,label=sec-6-10]{C and \swig}
 OK; so that was \CPP but I could have just as well have used C and \texttt{printf}.
A prime motivation for pairing \CPP and python is that both are OO languages.

\pause
For example, using C I have to tell \swig things like:
\lstset{language=C,label= ,caption= ,numbers=none}
\begin{lstlisting}
%newobject create_foo;
%delobject destroy_foo;

Foo *create_foo();
void destroy_foo(Foo *foo);
\end{lstlisting}
to handle creation/deletion of objects.
\end{frame}

\begin{frame}[fragile,label=sec-6-11]{\CPP and \swig}
 \lstset{language=c++,label= ,caption= ,numbers=none}
\begin{lstlisting}
  %module classes
  %include "std_string.i"
  
  %{
  #include <iostream>
  #include <string>
  %}
  
  %inline %{
  class Foo {
  public:
      Foo() { std::cout << "Hail, fair morning" << std::endl; }
      ~Foo() {
         std::cout << "It is a far, far better thing that I do now..."
                   << std::endl;
      }
  };
  %}
\end{lstlisting}

\pause
\lstset{language=Python,label= ,caption= ,numbers=none}
\begin{lstlisting}
>>> import classes
>>> x = classes.Foo()
Hail, fair morning
>>> del x
It is a far, far better thing that I do now...
>>> def tmp(): x = classes.Foo()
>>> tmp()
Hail, fair morning
It is a far, far better thing that I do now...
>>>
\end{lstlisting}

\pause (note that the destructor fired when \texttt{x} went out of scope)
\end{frame}

\begin{frame}[fragile,label=sec-6-12]{Proxy classes}
 \def\classes{\_classes}
In this case the \swig-generated \alert{classes.py} is more complicated.  It defines a python ``proxy'' class
\texttt{Foo}. For example, \texttt{\_\_init\_\_} calls \texttt{new\_Foo} in \alert{\classes.so}'s defined python
interface; which calls \texttt{\_wrap\_new\_Foo}; which calls the constructor \texttt{Foo()}:

\pause
\lstset{language=Python,label= ,caption= ,numbers=none}
\begin{lstlisting}
class Foo(_object):
    __swig_setmethods__ = {}
    __setattr__ = \
       lambda self, name, value: _swig_setattr(self, Foo, name, value)
    __swig_getmethods__ = {}
    __getattr__ = lambda self, name: _swig_getattr(self, Foo, name)
    __repr__ = _swig_repr
    def __init__(self): 
        this = _classes.@\color{red}new\_Foo@()
        try: self.this.append(this)
        except: self.this = this
    __swig_destroy__ = _classes.@\color{red}delete\_Foo@
    __del__ = lambda self : None;
Foo_swigregister = _classes.Foo_swigregister
Foo_swigregister(Foo)
\end{lstlisting}
\end{frame}

\begin{frame}[fragile,label=sec-6-13]{More complicated classes}
 That was fun.  Now for a slightly more interesting class:

\lstset{language=Python,label= ,caption= ,numbers=none}
\begin{lstlisting}
class Goo {
public:
   Goo() : i_(0), s_("") {}
   Goo(int i) : i_(i), s_("") {} Goo(std::string const& s) : i_(0), s_(s) {}
   int getI() const { return i_; }
   std::string getS() const { return s_; }
private:
   int i_;
   std::string s_;
};
\end{lstlisting}

\pause
\lstset{language=Python,label= ,caption= ,numbers=none}
\begin{lstlisting}
>>> import classes
>>> g = classes.Goo()
>>> print "%d \"%s\"" % (g.getI(), g.getS())
0 ""
>>> g = classes.Goo(12);  print "%d \"%s\"" % (g.getI(), g.getS())
12 ""
>>> g = classes.Goo("rhl");  print "%d \"%s\"" % (g.getI(), g.getS())
0 "rhl"
\end{lstlisting}
\pause

In this case, \texttt{Goo}'s proxy class has entries:
\lstset{language=Python,label= ,caption= ,numbers=none}
\begin{lstlisting}
    def getI(self): return _classes.@\color{red}Goo\_getI@(self)
    def getS(self): return _classes.@\color{red}Goo\_getS@(self)
\end{lstlisting}
\end{frame}

%\begin{frame}[fragile,label=sec-6-14]{What if I don't like accessor functions?}
% What about \texttt{struct}-style coding?
%
%\lstset{language=c++,label= ,caption= ,numbers=none}
%\begin{lstlisting}
%class Goo {
%public:
%    Goo() : i_(0), s_(""), x(-1) {}
%    Goo(int i) : i_(i), s_(""), x(-1) {}
%    Goo(std::string const& s) : i_(0), s_(s), x(-1) {}
%...
%    double x;
%};
%\end{lstlisting}

%\pause
%\lstset{language=Python,label= ,caption= ,numbers=none}
%\begin{lstlisting}
%>>> g.x
%-1.0
%>>> g.x = 10
%>>> g.x
%10.0
%\end{lstlisting}
%good
%\pause
%\lstset{language=Python,label= ,caption= ,numbers=none}
%\begin{lstlisting}
%>>> g.y = 1000
%>>> g.y
%1000
%\end{lstlisting}
%
%Not so good (\texttt{y} isn't a member of \texttt{Goo}).
%\end{frame}

%\begin{frame}[fragile,label=sec-6-15]{Preventing \swig from extending classes}
% The problem is that \texttt{Foo} is a python proxy class, and python permits
% anyone to add members such as \texttt{y} to a class.
%
%Fortunately, there's a solution:
%
%\lstset{language=c++,label= ,caption= ,numbers=none}
%\begin{lstlisting}
%%module goo
%%{
%#include <iostream>  
%#include "Goo.h"
%
%%}
%
%@\color{red}\%pythonnondynamic;@
%
%%include "Goo.h"
%\end{lstlisting}
%\pause
%and now:
%\lstset{language=Python,label= ,caption= ,numbers=none}
%\begin{lstlisting}
%>>> g = Goo(1)
%>>> g.y
%Traceback (most recent call last):
%  File "<stdin>", line 1, in <module>
%  File "goo.py", line 356, in <lambda>
%    __getattr__ = lambda self, name: _swig_getattr(self, Goo, name)
%  File "goo.py", line 54, in _swig_getattr
%    raise AttributeError(name)
%AttributeError: y
%\end{lstlisting}
%
%\pause

%that's better.
%\end{frame}

\begin{frame}[label=sec-6-16]{Extending the interface in python}
You can also add code to the python interface (it's also possible to add to the \CPP interface in
python using \swig \emph{directors}, but I've never tried).  It's not surprising that you can extend the
python layer when you recall the existence of proxy classes.
\end{frame}

\begin{frame}[fragile,label=sec-6-17]{Extending the interface in python}
 \lstset{language=C,label= ,caption= ,numbers=none}
\begin{lstlisting}
// -*- c++ -*-
%module goo

%{
#include <iostream>
#include "Goo.h"
%}
%pythonnondynamic;
%include "std_iostream.i"

%include "Goo.h"

%extend Goo {
   void printMe(std::ostream& os) {
      os << $self->getI() << " \"" << $self->getS() << "\"" << std::endl;
   }
   %pythoncode %{
   def __str__(self):
      return "%d \"%s\"" % (self.getI(), self.getS())
   %}
}
\end{lstlisting}
\pause
\lstset{language=Python,label= ,caption= ,numbers=none}
\begin{lstlisting}
>>> goo.Goo(12).printMe(goo.cout)
12 ""
>>> print goo.Goo('xxx')
0 "xxx"
\end{lstlisting}
\pause
You can easily imagine what the proxy class looks like.
\end{frame}

\begin{frame}[fragile,label=sec-6-18]{\swig v. the \emph{STL}}
 One reason to use \CPP is the Standard Template Library.

\lstset{language=c++,label= ,caption= ,numbers=none}
\begin{lstlisting}
// -*- c++ -*-
%module vector
%include "std_vector.i"

%{
#include <iostream>
#include <string>
#include <vector>

#include "Goo.h"
%}

%include "Goo.h"

%template(vectorGoo) std::vector<Goo>;
\end{lstlisting}

\pause
\lstset{language=Python,label= ,caption= ,numbers=none}
\begin{lstlisting}
>>> import vector
>>> v = vector.vectorGoo()
>>> v.push_back(vector.Goo(0))
>>> v.append(vector.Goo(1))
>>> len(v)
2
>>> v[1].getI()
1
\end{lstlisting}
\end{frame}

\begin{frame}[fragile,label=sec-6-19]{Why do I care?}
 So what? python already has vector-like lists.
\pause
Consider:
\lstset{language=C,label= ,caption= ,numbers=none}
\begin{lstlisting}
// -*- c++ -*-
%module vector2
%include "std_vector.i"

%{
#include <iostream>
#include <string>
#include <vector>
#include "Goo.h"
%}

%import "goo.i"

%template(vectorGoo) std::vector<Goo>;

%inline %{
   std::vector<Goo> *makeVectorGoo() {
      return new std::vector<Goo>;
   }

    void printGV(std::vector<Goo> const& gv) {
        for (auto ptr = gv.begin(); ptr != gv.end(); ++ptr) {
            std::cout << ptr->getI() << std::endl;
        }
    }
%}
\end{lstlisting}
\end{frame}

\begin{frame}[fragile,label=sec-6-20]{Using python's \texttt{list} with \texttt{std::vector} classes}
 \lstset{language=Python,label= ,caption= ,numbers=none}
\begin{lstlisting}
>>> import vector2 as vector; import goo
>>> v = vector.makeVectorGoo()          # equivalent to vector.VectorGoo()
>>> v.push_back(goo.Goo(0)); v.push_back(goo.Goo(1))
>>> vector.printGV(v)
0
1
>>> vv = [goo.Goo(10), goo.Goo(20)]
>>> vector.printGV(vv)
10
20
\end{lstlisting}

\pause

That's pretty nice;  we passed a python list to a \CPP function expecting a \texttt{std::vector<>}.
\end{frame}

\begin{frame}[fragile,label=sec-6-21]{\swig woes}
 \vspace*{-3mm}
If you put the \verb~%template~ line \emph{after} printGV, you'd get:
\lstset{language=C++,label= ,caption= ,numbers=none}
\begin{lstlisting}
Traceback (most recent call last):
  File "./vector2_demo.py", line 13, in <module>
    main()
  File "./vector2_demo.py", line 10, in main
    vector.printGV(vv)
TypeError: in method 'printGV', argument 1 of type 
                     'std::vector< Goo,std::allocator< Goo > > const &'
\end{lstlisting}

\pause
\swig needed to be told how to handle \verb~std::vector<Goo>~ before it saw the declaration

\pause
If you omit the \verb~%template~ line and try the \verb~push_back~ you get:
\lstset{language=c++,label= ,caption= ,numbers=none}
\begin{lstlisting}
Traceback (most recent call last):
  File "./vector2_demo.py", line 13, in <module>
    main()
  File "./vector2_demo.py", line 6, in main
    v.push_back(goo.Goo(0))
AttributeError: 'SwigPyObject' object has no attribute 'push_back'
\end{lstlisting}
\pause
whereas \verb~v.append(goo.Goo(666))~ produces:
\lstset{language=c++,label= ,caption= ,numbers=none}
\begin{lstlisting}
Traceback (most recent call last):
  File "./vector2_demo.py", line 15, in <module>
    main()
  File "./vector2_demo.py", line 7, in main
    v.append(goo.Goo(666))
SystemError: error return without exception set
\end{lstlisting}
\end{frame}

\begin{frame}[fragile,label=sec-6-22]{\swig \emph{typemaps}}
 Much of \swig is built around \emph{typemaps}. 
\pause
For example, when \swig is wrapping python the following maps are active
\lstset{language=C,label= ,caption= ,numbers=none}
\begin{lstlisting}
/* Convert from Python --> C */
%typemap(in) int {
    $1 = PyInt_AsLong($input);
}

/* Convert from C --> Python */
%typemap(out) int {
    $result = PyInt_FromLong($1);
}
\end{lstlisting}
These are essentially macros that generate \CPP in the \alert{\wrap.cc} file.

\pause
To generate perl bindings, we use a different typemap:
\lstset{language=C,label= ,caption= ,numbers=none}
\begin{lstlisting}
/* Convert from Perl --> C */
%typemap(in) int {
   $1 = SvIV($input);
}
\end{lstlisting}
\end{frame}

\begin{frame}[fragile,label=sec-6-23]{\swig \emph{typemaps}}
 For example, given
\lstset{language=C,label= ,caption= ,numbers=none}
\begin{lstlisting}
int foo(int x, int y);
\end{lstlisting}
\pause
\swig writes something like:
\lstset{language=C,label= ,caption= ,numbers=none}
\begin{lstlisting}
PyObject *wrap_foo(PyObject *self, PyObject *args) {
   int arg1, arg2;
   int result;
   PyObject *obj1, *obj2;
   PyObject *resultobj;

   if (!PyArg_ParseTuple("OO:foo", &obj1, &obj2)) return NULL;

   @\color{red}arg1 = PyInt\_AsLong(obj1);@
   @\color{red}arg2 = PyInt\_AsLong(obj2);@

   result = foo(arg1);

   @\color{red}resultobj = PyInt\_FromLong(result);@

   return resultobj;
}
\end{lstlisting}
\pause
You can write your own typemaps if you have special needs, but generally speaking the casual \swig user
doesn't need to learn these arcana. 
\end{frame}

\begin{frame}[fragile,label=sec-6-24]{applying typemaps}
 I can specify that a typemap only be applied to an argument with a particular name:
\lstset{language=C,label= ,caption= ,numbers=none}
\begin{lstlisting}
  %typemap(in) int @\color{red}positive@ {
      $1 = PyInt_AsLong($input);
      if ($1 <= 0) {
         SWIG_exception(SWIG_ValueError,"Expected positive value.");
      }
  }
\end{lstlisting}
\pause
(It would be better to use a \texttt{\%typemap(check)} typemap).

\pause
Unfortunately, my routine looks like:
\lstset{language=C,label= ,caption= ,numbers=none}
\begin{lstlisting}
int *newArray(int n);
\end{lstlisting}
How do I check that \texttt{n} is positive?
\pause
The solution is to ask \swig to \texttt{\%apply} my \texttt{int positive} map to \texttt{n}:
\lstset{language=C,label= ,caption= ,numbers=none}
\begin{lstlisting}
%apply int positive { int n };
\end{lstlisting}
and now \texttt{newArray}'s argument is checked.
\end{frame}

\begin{frame}[fragile,label=sec-6-30]{\BP \emph{v}. \texttt{cython} \emph{v}. \swig}
 There appear to be three viable technologies to wrap \CPP and python: \BP, \texttt{cython}, and \swig.
  Which should you use?
\pause
\begin{block}{In simple case \swig is less work; you tell it about your \alert{.h} file and it generates the full interface. \pause In many complicated cases this is still true, but when it fails, it can be very confusing.}
\pause
\end{block}
\begin{block}{With \texttt{cython} you have to say what you want to wrap, and \texttt{cython} generates the interface \pause}
\end{block}
\begin{block}{You have to tell \BP exactly what you want to expose, and how to do so --- but \BP then does exactly what you told it to do}
\pause
\end{block}
\begin{block}{Ultimately the decision is a matter of taste. \pause I prefer \texttt{cython}.}
\end{block}
\end{frame}
% Emacs 24.5.1 (Org mode 8.2.7c)

\section*{CPython}
\label{sec-7}
\begin{frame}[label=sec-7-1]{Using the C API, \emph{CPython}}
\begin{block}{Python's C API}
\url{http://docs.python.org/extending/index.html\#extending-index}
\url{http://docs.python.org/c-api}
\end{block}
\end{frame}

\end{document}
